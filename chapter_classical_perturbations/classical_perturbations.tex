\chapter{Classical perturbations to the kinetically dominated universe}
\label{chap:cls}


To first order, the $i$-$j$, $i$-$0$ and $0$-$0$ Einstein equations read:
\begin{align}
  \Phi &= \Psi 
  \label{eqn:clp:Eij} \\
  0 &= -\dot{\phi}\:\delta\phi  + 2 H \Phi + 2 \dot{\Phi} 
  \label{eqn:clp:Ei0}\\
  0 &= \left(6H^2-\dot{\phi}^2 + 2\frac{k^2}{a^2}\right)\Phi  + \left( -3H\dot{\phi} - \ddot{\phi} \right)\delta\phi + \dot{\phi}\delta\dot{\phi} +  6 H \dot{\Phi},
  \label{eqn:clp:E00}
\end{align}
where we have absorbed the explicit potential dependence into $\ddot{\phi}$ terms. 
One may rearrange these to gain second order equations in $\Phi$ or $\mathcal{R}$:
\begin{align}
  0 &= \ddot{\Phi} + \left( H - 2 \frac{\ddot{\phi}}{\dot{\phi}} \right) \dot{\Phi} + \left( \frac{k^2}{a^2}-\frac{1}{\m^2}\dot{\phi}^2 - 2\frac{\ddot{\phi}}{\dot{\phi}}H \right)\Phi, \\
  0 &= \ddot{\mathcal{R}} + \left( \frac{\dot{\phi}^2}{\m^2H} + 3H + 2\frac{\ddot{\phi}}{\dot{\phi}} \right)\dot{\mathcal{R}} + \frac{k^2}{a^2}\mathcal{R}, \qquad \mathcal{R} = \Psi - \frac{H}{\dot{\phi}}\delta\phi
\end{align}
There is of course also a second order equation purely in $\delta\phi$, but this is particularly unappetising, and given that we have an explicit solution for it in~\eqref{eqn:clp:Ei0}, we shall not state it here.
Technically, these equations are derived in the Newtonian gauge ($E=B=0$), but since everything here is manifestly gauge invariant one may interpret all of the above equations as the relations in the equivalent gauge invariant variables.

If we apply the kinetically dominated solutions to the background variables, the perturbed variables are soluble with Bessel functions. There are two constants of integration, in the above equations. Examining the limit as $t\to 0$, one finds that all of the variables have a ``freezing mode'' and a ``growing mode'', i.e:
\begin{equation}
  \Phi\sim \delta\phi \sim \mathcal{R} = A + B t^{-4/3} \qquad \text{as } t\rightarrow 0,
\end{equation}
where $A$ and $B$ are constants of integration.
The growing mode in $\log t$ technically violates the initial perturbative assumption. It may be the case that a more sophisticated expansion out of the big bang is required.

We may rearrange equations~\eqref{eqn:clp:Ei0} and~\eqref{eqn:clp:E00} as:
\begin{align}
  \frac{d}{dt}
  \left(
  \begin{array}{c}
    \delta \phi \\
    \Phi
  \end{array}
  \right)
  =
  \left(%
  \begin{array}{cc}
    \frac{\ddot{\phi}}{\dot{\phi}} & \frac{1}{\dot{\phi}}\left( \dot{\phi}^2 - 2\frac{k^2}{a^2} \right) \\
    \frac{1}{2}\dot{\phi} & -H 
  \end{array}
  \right)
  \left(%
  \begin{array}{c}
    \delta \phi \\
    \Phi
  \end{array}
  \right)
\end{align}


Is there an alternative variable which freezes out in the kinetically dominated phase? Yes, there must be. What is it? What is the best way to find out what it is? Is it gauge invariant?

\ifdefined\lightweight
\else
\begin{figure}
  \begin{tikzpicture}
  \begin{axis}[%
      xlabel=$m\phi$,
      ylabel=$\dot{\phi}$,
      axis equal,
      width=\textwidth,
      xtick={-1,1},
      ytick={-1,1},
      xticklabels={{$-\m$}, {$\m$}},
      yticklabels={{$-\m$}, {$\m$}},
      xmin = -1.45,
      xmax =  1.45,
      ymin = -1.45,
      ymax =  1.45,
      legend cell align=right,
    ]

    \foreach \n  in {1,2,...,82} {%
      \addplot[thin,forget plot] table[] {chapter_classical_perturbations/data/swirl\n.dat}; 
    }

    \filldraw [fill=black] (axis cs:0,0) circle [radius=4.5pt];


    \fill [opacity=0.2]  
    (axis cs: {sqrt(2)*cos(45)},{sqrt(2)*sin(45)}) 
    arc [
      radius=transformdirectionx(sqrt(2)),
      start angle=45, 
      end angle=-45
    ] --
    (axis cs: {sqrt(2)*cos(-225)},{sqrt(2)*sin(-225)}) 
    arc [
      radius=transformdirectionx(sqrt(2)),
      start angle=-225, 
      end angle=-135
    ];
    \addlegendimage{area legend, fill=black,opacity=0.2}
    \addlegendentry{Inflating}



    \fill [opacity=0.6]  
    (axis cs: {sqrt(2)*cos(85)},{sqrt(2)*sin(85)}) 
    arc [
      radius=transformdirectionx(sqrt(2)),
      start angle=85, 
      end angle=95
    ] --
    (axis cs: {sqrt(2)*cos(-85)},{sqrt(2)*sin(-85)}) 
    arc [
      radius=transformdirectionx(sqrt(2)),
      start angle=-85, 
      end angle=-95
    ];
    \addlegendimage{area legend, fill=black,opacity=0.6}
    \addlegendentry{KD}




    \draw[dotted] (axis cs: 0,0) circle [radius=transformdirectionx(sqrt(2))];

    \addlegendimage{dotted}
    \addlegendentry{$E=\m$}

    %\addlegendimage{red,mark=square*}
    \node[anchor=north west] at (rel axis cs:0,1) {$m=0.5\m$};

  \end{axis}



\end{tikzpicture}

  \begin{tikzpicture}
  \begin{axis}[%
      xlabel=$m\phi$,
      ylabel=$\dot{\phi}$,
      axis equal,
      width=\textwidth,
      xtick={-1,1},
      ytick={-1,1},
      xticklabels={{$-\m$}, {$\m$}},
      yticklabels={{$-\m$}, {$\m$}},
      xmin = -1.45,
      xmax =  1.45,
      ymin = -1.45,
      ymax =  1.45,
      legend cell align=right,
    ]

    \foreach \n  in {1,2,...,82} {%
      \addplot[thin,forget plot] table[] {chapter_classical_perturbations/data/true\n.dat}; 
    }

    \filldraw [fill=black] (axis cs:0,0) circle [radius=2pt];


    \fill [opacity=0.2]  
    (axis cs: {sqrt(2)*cos(45)},{sqrt(2)*sin(45)}) 
    arc [
      radius=transformdirectionx(sqrt(2)),
      start angle=45, 
      end angle=-45
    ] --
    (axis cs: {sqrt(2)*cos(-225)},{sqrt(2)*sin(-225)}) 
    arc [
      radius=transformdirectionx(sqrt(2)),
      start angle=-225, 
      end angle=-135
    ];
    \addlegendimage{area legend, fill=black,opacity=0.2}
    \addlegendentry{Inflating}



    \fill [opacity=0.6]  
    (axis cs: {sqrt(2)*cos(85)},{sqrt(2)*sin(85)}) 
    arc [
      radius=transformdirectionx(sqrt(2)),
      start angle=85, 
      end angle=95
    ] --
    (axis cs: {sqrt(2)*cos(-85)},{sqrt(2)*sin(-85)}) 
    arc [
      radius=transformdirectionx(sqrt(2)),
      start angle=-85, 
      end angle=-95
    ];
    \addlegendimage{area legend, fill=black,opacity=0.6}
    \addlegendentry{KD}




    \draw[dotted] (axis cs: 0,0) circle [radius=transformdirectionx(sqrt(2))];

    \addlegendimage{dotted}
    \addlegendentry{$E=\m$}

    %\addlegendimage{red,mark=square*}
    \node[anchor=north west] at (rel axis cs:0,1) {$m=0.1\m$};

  \end{axis}



\end{tikzpicture}

  \caption{Planckian initial conditions}
\end{figure}

\begin{figure}
  \begin{tikzpicture}
  \begin{axis}[%
      xlabel=$m\phi$,
      ylabel=$\dot{\phi}$,
      axis equal,
      width=\textwidth,
      xtick={-1,1},
      ytick={-1,1},
      xticklabels={{$-\m$}, {$\m$}},
      yticklabels={{$-\m$}, {$\m$}},
      xmin = -1.45,
      xmax =  1.45,
      ymin = -1.45,
      ymax =  1.45,
      legend cell align=right,
    ]

    \foreach \n  in {1,2,...,82} {%
      \addplot[thin,forget plot] table[] {chapter_classical_perturbations/data/swirl_alt\n.dat}; 
    }

    %\filldraw [fill=black] (axis cs:0,0) circle [radius=3.5pt];


    \fill [opacity=0.2]  
    (axis cs: {sqrt(2)*cos(45)},{sqrt(2)*sin(45)}) 
    arc [
      radius=transformdirectionx(sqrt(2)),
      start angle=45, 
      end angle=-45
    ] --
    (axis cs: {sqrt(2)*cos(-225)},{sqrt(2)*sin(-225)}) 
    arc [
      radius=transformdirectionx(sqrt(2)),
      start angle=-225, 
      end angle=-135
    ];
    \addlegendimage{area legend, fill=black,opacity=0.2}
    \addlegendentry{Inflating}



    \fill [opacity=0.6]  
    (axis cs: {sqrt(2)*cos(85)},{sqrt(2)*sin(85)}) 
    arc [
      radius=transformdirectionx(sqrt(2)),
      start angle=85, 
      end angle=95
    ] --
    (axis cs: {sqrt(2)*cos(-85)},{sqrt(2)*sin(-85)}) 
    arc [
      radius=transformdirectionx(sqrt(2)),
      start angle=-85, 
      end angle=-95
    ];
    \addlegendimage{area legend, fill=black,opacity=0.6}
    \addlegendentry{KD}




    \draw[dotted] (axis cs: 0,0) circle [radius=transformdirectionx(sqrt(2))];

    \addlegendimage{dotted}
    \addlegendentry{$E=\m$}

    %\addlegendimage{red,mark=square*}
    \node[anchor=north west] at (rel axis cs:0,1) {$m=0.5\m$};

  \end{axis}



\end{tikzpicture}

  \begin{tikzpicture}
  \begin{axis}[%
      xlabel=$m\phi$,
      ylabel=$\dot{\phi}$,
      axis equal,
      width=\textwidth,
      xtick={-1,1},
      ytick={-1,1},
      xticklabels={{$-\m$}, {$\m$}},
      yticklabels={{$-\m$}, {$\m$}},
      xmin = -1.45,
      xmax =  1.45,
      ymin = -1.45,
      ymax =  1.45,
      legend cell align=right,
    ]

    \foreach \n  in {1,2,...,82} {%
      \addplot[thin] table[] {chapter_classical_perturbations/data/true_alt\n.dat}; 
    }

    %\filldraw [fill=black] (axis cs:0,0) circle [radius=3.5pt];


    \fill [opacity=0.2]  
    (axis cs: {sqrt(2)*cos(45)},{sqrt(2)*sin(45)}) 
    arc [
      radius=transformdirectionx(sqrt(2)),
      start angle=45, 
      end angle=-45
    ] --
    (axis cs: {sqrt(2)*cos(-225)},{sqrt(2)*sin(-225)}) 
    arc [
      radius=transformdirectionx(sqrt(2)),
      start angle=-225, 
      end angle=-135
    ];
    \addlegendimage{area legend, fill=black,opacity=0.2}
    \addlegendentry{Inflating}



    \fill [opacity=0.6]  
    (axis cs: {sqrt(2)*cos(85)},{sqrt(2)*sin(85)}) 
    arc [
      radius=transformdirectionx(sqrt(2)),
      start angle=85, 
      end angle=95
    ] --
    (axis cs: {sqrt(2)*cos(-85)},{sqrt(2)*sin(-85)}) 
    arc [
      radius=transformdirectionx(sqrt(2)),
      start angle=-85, 
      end angle=-95
    ];
    \addlegendimage{area legend, fill=black,opacity=0.6}
    \addlegendentry{KD}




    \draw[dotted] (axis cs: 0,0) circle [radius=transformdirectionx(sqrt(2))];

    \addlegendimage{dotted}
    \addlegendentry{$E=\m$}

    %\addlegendimage{red,mark=square*}
    \node[anchor=north west] at (rel axis cs:0,1) {$m=0.1\m$};

  \end{axis}



\end{tikzpicture}

  \caption{Post-inflationary initial conditions}
\end{figure}
\fi
