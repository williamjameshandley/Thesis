\chapter{Classical perturbations to the kinetically dominated universe}
\label{chap:cls}

\section{The perturbed universe}
The real universe is not smooth, despite how much of a good approximation that might be. Using the FRW metric as a 0\textsuperscript{th} order approximation, we may expand about these smooth solutions. In general then, we write each quantity as:
\begin{equation}
  X(t,\vx) = X(t) + \delta X(t,\vx)
  \label{eqn:cos:expansion}
\end{equation}
Since in the early universe the perturbations were small $\delta X \ll X$, we may expand all equations to linear order to very high accuracy.

\begin{table}
  \centering
\begin{tabular}{lll}
 \toprule
  Symbol & Definition & Properties \\
 \midrule
 \midrule
 $\Phi$ & lapse & scalar\\
 $B_i$ & shift & vector\\
 $\Psi$ & (spatial) curvature perturbation  & scalar\\
 $E_{ij}$ & (spatial) shear (3-tensor) & symmetric \& traceless tensor\\
 \bottomrule
\end{tabular}
\caption{Definitions of terms in the perturbed FRW metric}\label{tab:cos:perturbed_metric}
\end{table}


A general perturbation of the FRW metric will take the form:
\begin{equation}
  ds^2 = (1+2\Phi)dt^2 -2a B_i dx^i dt  -a{(t)}^2 \left[ \left( 1 - 2 \Psi \right)\delta_{ij} + 2E_{ij} \right] dx^i dx^j,
  \label{eqn:cos:FRW_perturb}
\end{equation}
where the various terms in the above expression are defined in Table~\ref{tab:cos:perturbed_metric}, and $\Phi,B_i,\Psi,E_{ij}\ll1$. Perturbing the matter content will yield a perturbed stress energy tensor, and to first order the Einstein equation~\eqref{eqn:cos:einsteins_equations} becomes:
\begin{equation}
 \m^2 \delta G_{\mu\nu} = \delta T_{\mu\nu}.
  \label{eqn:cos:einsteins_equations_perturb}
\end{equation}


\subsection{Fourier modes and SVT decomposition}
In general, the Einstein equations~\eqref{eqn:cos:einstein_tensor} are non-linear, second order, partial differential equations and are therefore extremely challenging to solve. However, the symmetry of the unperturbed universe makes linear perturbations much simpler. 
\begin{itemize}
  \item Translational invariance of the unperturbed universe means that the Fourier modes of the perturbations do not interact, simply turning spatial derivatives into multiples of wavevectors. 
  \item Further the rotational invariance of  the unperturbed universe means that when the 3-vector and 3-tensor perturbations are decomposed into helicity modes, these helicity modes are also non-interacting.
\end{itemize}
Formally, a general field can be written in Fourier modes as:
\begin{align}
  \delta X(t,\vk) &= \int d^3\vx\: \delta X(t,\vx) e^{-i\vk \cdot \vx},\\
  \delta X(t,\vx) &= \int \frac{d^3\vk}{{(2\pi)}^3}\: \delta X(t,\vx) e^{i\vk \cdot \vx}.
\end{align}
A vector field $V_i$ and a symmetric, traceless tensor field $T_{ij}$ can be decomposed into helicity modes as:
\begin{align}
  V_i =& \partial_i V + V_i,   \nonumber\\
  &(\partial^k V_k=0) \\
  T_{ij} =& (\partial_i\partial_j - \frac{\delta_{ij}}{3}\partial^k\partial_k)T + \frac{1}{2}(\partial_i T_j + \partial_j T_i) + T_{ij} \nonumber\\ 
  &(\partial^k T_{ki} = \partial^k T_k = 0),
\end{align}
where $V$ and $T$ are helicity scalars, $V_i$ and $T_i$ are divergenceless 3-vectors and $T_{ij}$ is a divergenceless, symmetric, traceless 3-tensor\footnote{Note the overloaded notation: $V_i$ and $T_{ij}$ mean different things on either side of each equation.}.

We may therefore decompose the 
\subsection{Gauge invariant equations}



\subsection{Gauge choice}
Observant readers will have noted that these equations are incomplete. For example, the scalar equations~\eqref{eqn:cos:scalar_1}--\eqref{eqn:cos:scalar_4} constitute four equations in seven perturbed variables. Now, $\delta\rho$, $\delta P$ and $\delta\Sigma$ will be supplemented with equations of state (typically $\delta\rho = w\:\delta P$, $\delta\Sigma=0$), but that still leaves two degrees of freedom to be fixed.

This lack of constraint arises from the fact that the split into background and perturbation $X=X+\delta X$ implied by~\eqref{eqn:cos:expansion} is more subtle than first appears. 

Gauge\footnote{$\delta X$ is defined as the difference between the value $X$ has in the physical (perturbed) spacetime, and the value $X$ has in the background (unperturbed) spacetime. This can only be done if there is a prescription for identifying points between the two spacetimes, and in the language of differential geometry, this is termed a {\em gauge choice}.}

\begin{table}
  \centering
\begin{tabular}{ll}
 \toprule
  Name & Definition \\
 \midrule
 \midrule
 Synchronous & $\Phi=B=0$ \\
 Newtonian & $B=E=0$ \\
 Uniform density & $\delta\rho=0$ and e.g.\ $E=0$ \\
 Comoving & $\delta q = E = 0$ \\
 Spatially-flat & $\Psi=E=0$ \\
 \bottomrule
\end{tabular}
\caption{Popular gauge choices}\label{tab:cos:gauge_choice}
\end{table}
Gauge transformations
\begin{align}
  t &\rightarrow t + \delta t  \\
  x^i &\rightarrow x^i  + \delta x^i \nonumber\\
  &=   x^i + \partial^i \delta x + \delta x^i
\end{align}
\begin{align}
  \Phi &\rightarrow \Phi - \delta \dot{t} &
  E &\rightarrow E - \delta x \nonumber \\
  \Psi &\rightarrow \Psi +H \delta t &
  E_i & \rightarrow E_i - \delta x_i\nonumber \\
  B &\rightarrow B + \delta t/a - a\delta \dot{x} &
  E_{ij} &\rightarrow E_{ij}\nonumber \\
  B_i &\rightarrow  B_i + a \delta\dot{x}_i &
\end{align}

Gauge independent variables:
\begin{align}
  \Phi_B &=  \Phi - \frac{d}{dt}(a^2[\dot{E}-B/a]) \\
  \Psi_B &=  \Psi + Ha^2[\dot{E}-B/a] \\
\end{align}

Einstein tensor:
\begin{align}
  G^{0}_{0} &= 3 H^2 \\
  G^{i}_{j} &= \left( 2\dot{H} + 3H^2 \right)\delta^{i}_{j}
\end{align}

Perturbed Einstein tensor for scalar parts, $E=B=0$:
\begin{align}
  \delta G^{0}_{0} &= 2\frac{\nabla^2}{a^2} \Psi -6H\dot{\Psi} -6H^2\Phi\\
  \delta G^{0}_{i} &= 2\nabla_i(\dot{\Psi} + \Phi H) \\
  \delta G^{i}_{j} &= -\left( 2 \Phi H^2 +4 \Phi \dot{H} + 2\dot{\Phi}H +6H\dot{\Psi} +2 \ddot{\Psi} + \frac{\nabla^2}{a^2}[\Phi-\Psi] \right)\delta^{i}_{j} + \frac{\nabla^{i}\nabla_{j}}{a^2}[\Phi-\Psi]
\end{align}
