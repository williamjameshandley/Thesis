\chapter{Classical perturbations to the kinetically dominated universe}
\label{chap:cls}

\section{The perturbed universe}
The real universe is not smooth, despite how much of a good approximation that might be. Using the FRW metric as a 0\textsuperscript{th} order approximation, we may expand about these smooth solutions. In general then, we write each quantity as:
\begin{equation}
  X(t,\vx) = X(t) + \delta X(t,\vx)
  \label{eqn:cos:expansion}
\end{equation}
Since in the early universe the perturbations were small $\delta X \ll X$, we may expand all equations to linear order to very high accuracy.

\begin{table}
  \centering
\begin{tabular}{lll}
 \toprule
  Symbol & Definition & Properties \\
 \midrule
 \midrule
 $\Phi$ & lapse & scalar\\
 $B_i$ & shift & vector\\
 $\Psi$ & (spatial) curvature perturbation  & scalar\\
 $E_{ij}$ & (spatial) shear (3-tensor) & symmetric \& traceless tensor\\
 \bottomrule
\end{tabular}
\caption{Definitions of terms in the perturbed FRW metric}\label{tab:cos:perturbed_metric}
\end{table}


A general perturbation of the FRW metric will take the form:
\begin{equation}
  ds^2 = (1+2\Phi)dt^2 -2a B_i dx^i dt  -a{(t)}^2 \left[ \left( 1 - 2 \Psi \right)\delta_{ij} + E_{ij} \right] dx^i dx^j,
  \label{eqn:cos:FRW_perturb}
\end{equation}
where the various terms in the above expression are defined in Table~\ref{tab:cos:perturbed_metric}, and $\Phi,B_i,\Psi,E_{ij}\ll1$. Perturbing the matter content will yield a perturbed stress energy tensor, and to first order the Einstein equation~\eqref{eqn:cos:einsteins_equations} becomes:
\begin{equation}
 \m^2 \delta G_{\mu\nu} = \delta T_{\mu\nu}.
  \label{eqn:cos:einsteins_equations_perturb}
\end{equation}


\subsection{Fourier modes and SVT decomposition}
In general, the Einstein equations~\eqref{eqn:cos:einstein_tensor} are non-linear, second order, partial differential equations and are therefore extremely challenging to solve. However, the symmetry of the unperturbed universe makes linear perturbations much simpler. 
\begin{itemize}
  \item Translational invariance of the unperturbed universe means that the Fourier modes of the perturbations do not interact, simply turning spatial derivatives into multiples of wavevectors. 
  \item Further the rotational invariance of  the unperturbed universe means that when the 3-vector and 3-tensor perturbations are decomposed into helicity modes, these helicity modes are also non-interacting.
\end{itemize}
Formally, a general field can be written in Fourier modes as:
\begin{align}
  \delta X(t,\vk) &= \int d^3\vx\: \delta X(t,\vx) e^{-i\vk \cdot \vx},\\
  \delta X(t,\vx) &= \int \frac{d^3\vk}{{(2\pi)}^3}\: \delta X(t,\vx) e^{i\vk \cdot \vx}.
\end{align}
A vector field $V_i$ and a symmetric, traceless tensor field $T_{ij}$ can be decomposed into helicity modes as:
\begin{align}
  V_i =& \partial_i V + V_i,   \nonumber\\
  &(\partial^k V_k=0) \\
  T_{ij} =& (\partial_i\partial_j - \frac{\delta_{ij}}{3}\partial^k\partial_k)T + \frac{1}{2}(\partial_i T_j + \partial_j T_i) + T_{ij} \nonumber\\ 
  &(\partial^k T_{ki} = \partial^k T_k = 0),
\end{align}
where $V$ and $T$ are helicity scalars, $V_i$ and $T_i$ are divergenceless 3-vectors and $T_{ij}$ is a divergenceless, symmetric, traceless 3-tensor\footnote{Note the overloaded notation: $V_i$ and $T_{ij}$ mean different things on either side of each equation.}.

\subsection{Gauge invariant equations}



\subsection{Einstein equations}
We now have the ingredients to form the first-order Einstein equations of the perturbed universe. Solving equation~\eqref{eqn:cos:einsteins_equations} using the perturbed forms~\eqref{eqn:cos:FRW_perturb}--\eqref{eqn:cos:sigma_perturb} with all variables decomposed into Fourier and helicity modes yields separate equations for scalars vectors and tensors listed below.
\subsubsection{Scalars}
\begin{align}
  3H\left( \dot{\Psi} + H \Phi  \right) + \frac{k^2}{a^2}\left[ \Psi + H\left( a^2\dot{E}-aB \right) \right] &= -4\pi G \delta\rho
  \label{eqn:cos:scalar_1} \\
  \dot{\Psi} + H \Phi &= -4\pi G \delta q 
  \label{eqn:cos:scalar_2}\\
  \dot{\Psi} + 3 H \dot{\Psi} + H \dot{\Phi} + \left( 3H^2 + 2\dot{H} \right)\Phi &= 4\pi G \left( \delta\rho - \frac{2}{3}k^2\delta\Sigma \right)
  \label{eqn:cos:scalar_3}\\
  \left( \partial_t + 3H \right)\left( \dot{E}-B/a \right) + \frac{\Psi-\Phi}{a^2} &= 8\pi G \delta\Sigma
  \label{eqn:cos:scalar_4}
\end{align}
\subsubsection{Vectors}
\begin{align}
  \delta \dot{q}_i + 3H \delta q_i &= k^2 \delta \Sigma_i \\
  k^2\left( \dot{E}_i + B_i/a  \right) &= 16\pi G \delta q_i
\end{align}


\subsubsection{Tensors}
\begin{align}
  S&=\ddot{h} + 3H \dot{h} + \frac{k^2}{a^2}h \\
  E_{ij} &= h(t) e_{ij}^{(+,\times)}(x), \qquad 
  \delta\Sigma_{ij} = S(t) e_{ij}^{(+,\times)}(x), \qquad 
  \nabla^2e_{ij} = -k^2 e_{ij}
\end{align}

\subsection{Gauge choice}
Observant readers will have noted that these equations are incomplete. For example, the scalar equations~\eqref{eqn:cos:scalar_1}--\eqref{eqn:cos:scalar_4} constitute four equations in seven perturbed variables. Now, $\delta\rho$, $\delta P$ and $\delta\Sigma$ will be supplemented with equations of state (typically $\delta\rho = w\:\delta P$, $\delta\Sigma=0$), but that still leaves two degrees of freedom to be fixed.

This lack of constraint arises from the fact that the split into background and perturbation $X=X+\delta X$ implied by~\eqref{eqn:cos:expansion} is more subtle than first appears. 

Gauge\footnote{$\delta X$ is defined as the difference between the value $X$ has in the physical (perturbed) spacetime, and the value $X$ has in the background (unperturbed) spacetime. This can only be done if there is a prescription for identifying points between the two spacetimes, and in the language of differential geometry, this is termed a {\em gauge choice}.}

\begin{table}
  \centering
\begin{tabular}{ll}
 \toprule
  Name & Definition \\
 \midrule
 \midrule
 Synchronous & $\Phi=B=0$ \\
 Newtonian & $B=E=0$ \\
 Uniform density & $\delta\rho=0$ and e.g.\ $E=0$ \\
 Comoving & $\delta q = E = 0$ \\
 Spatially-flat & $\Psi=E=0$ \\
 \bottomrule
\end{tabular}
\caption{Popular gauge choices}\label{tab:cos:gauge_choice}
\end{table}
Scalar gauge transformations:
\begin{align}
      t &\rightarrow t + \delta t 
  & x^i &\rightarrow x^i + \partial^i \delta x  \\
   \Phi &\rightarrow \Phi - \delta \dot{t}
  &\Psi &\rightarrow \Psi +H \delta t  \\
      B &\rightarrow B + \delta t/a - a\delta \dot{x}
  &   E &\rightarrow E - \delta x 
\end{align}
Vector gauge transformations:
\begin{align}
  x^i &\rightarrow x^i + \delta x^i, \qquad \partial^k\delta x_k = 0   \\
  B_i &\rightarrow  B_i + a \delta\dot{x}_i \\
  E_i & \rightarrow E_i - \delta x_i
\end{align}

We will also have an equation of state, which links $\delta\rho$, $\delta P$ and $\delta\Sigma$, typically $\delta\rho = w \delta P$, $\delta\Sigma=0$.

Five equations in Seven variables $\{\delta\rho, \delta P, \Phi, \Psi, B, E, \delta q, \delta\Sigma\}$. But gauge transformations can remove a further two of them.

Gauge independent variables:

\subsection{Metric perturbations}

\subsection{Matter perturbations}
Perturbing the matter content will yield:
\begin{align}
  \rho &= \rho + \delta\rho 
  \label{eqn:cos:rho_perturb}\\
  P &= P + \delta P 
  \label{eqn:cos:P_perturb}\\
  u^\mu &= u^\mu + \delta u^\mu 
  \label{eqn:cos:u_perturb}\\
  \Sigma^{\mu\nu} &= \Sigma^{\mu\nu}+ \delta\Sigma^{\mu\nu}
  \label{eqn:cos:sigma_perturb}
\end{align}
We may constrain $\delta u^\mu$ by the requirement that $u^2=1$, so:
\begin{equation}
  \delta u^\mu = (-\Phi,\delta u^i).
\end{equation}
Similarly, the anisotropic stress is constrained such that only the $\delta\Sigma^{ij}$ terms are non-zero.

When combining separate components, density, pressure and anisotropic stress add linearly, but velocities do not. It is convenient to define the 3-momentum density:
\begin{equation}
  \delta q^i = (\rho+P)/a^2 \delta u^i,
\end{equation}
which is additive.



\begin{align}
  -\zeta &\equiv \Psi + \frac{H}{\dot{\rho}}\delta\rho
  \label{eqn:cos:gauge_zeta}\\
  \mathcal{R} &\equiv \Psi + \frac{H}{P+\rho}\delta q
  \label{eqn:cos:gauge_R}\\
  Q &\equiv \delta q + \frac{\dot{\phi}}{H}\Psi
  \label{eqn:cos:gauge_Q}\\
\end{align}

