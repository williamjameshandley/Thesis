\chapter{Inflationary Cosmology}
\label{chap:cos}

\section{Introduction}
\label{sec:cos:intro}

\section{Conventions}
\label{sec:cos:conventions}


\begin{enumerate}
  \item I work using a metric with a positive signature $(+,-,-,-)$.
  \item Fourier transforms are written with an overloaded notation, and defined so that Fourier synthesis carries the factors of $2\pi$:
    \begin{equation}
      f(\vk) = \int_{-\infty}^{\infty} d^3\vx\: e^{-i \vk\cdot\vx} f(\vx) \qquad f(\vx) = \int_{-\infty}^{\infty} \frac{d^3\vk}{(2\pi)^3}\: e^{i \vx\cdot\vk} f(\vk).
      \label{eqn:cos:fourier_transform}
    \end{equation}
  \item I work in natural units so that 
    \begin{equation}
      c = \hbar = G = k_B = 1,
      \label{eqn:cos:natural_units}
    \end{equation}
    but retain the reduced Planck mass for clarity:
    \begin{equation}
      \m = \frac{1}{8\pi G}.
      \label{eqn:cos:reduced_planck_mass}
    \end{equation}
    
\end{enumerate}


\section{Einstein's gravity}
\label{sec:cos:einsteins_gravity}

Einstein's gravity can be effectively summarised using the Einstein-Hilbert action formalism. An action $S$ is written as a general relativistic integral over a Lagrangian density $\mathcal{L}$:\footnote{This should not be confused with a likelihood, also denoted with $\lik$.}
\begin{equation}
  S = \int d^4 x \sqrt{|g|} R
  \label{eqn:cos:action}
\end{equation}
where $g=\det(g_{\mu\nu})$ is the determinant of the metric, and $R$ is the Ricci scalar, defined by
\begin{align}
  R &= R^\mu_\mu \label{eqn:cos:ricci_scalar_def} \\
  R_{\mu\nu} &= R^\rho_{\mu\rho\nu} \label{eqn:cos:ricci_tensor_def} \\
  R_{\mu\nu} &= R^\rho_{\mu\rho\nu} \label{eqn:cos:riemann_tensor_def} \\
  R^\rho_{\sigma a\nu} &= \partial_\mu\Gamma^\rho{}_{\nu\sigma}
    - \partial_\nu\Gamma^\rho{}_{\mu\sigma}
    + \Gamma^\rho{}_{\mu\lambda}\Gamma^\lambda{}_{\nu\sigma}
    - \Gamma^\rho{}_{\nu\lambda}\Gamma^\lambda{}_{\mu\sigma}
\end{align}

We define the action:
\begin{equation}
  S = S_G + S_m
  \label{eqn:cos:action}
\end{equation}
where
\begin{equation}
  S_G = \int \frac{1}{2}
  \label{}
\end{equation}

\section{Background}
\label{sec:background}

\clearpage{}

We denote a general action via
\begin{equation}
  S_I = \int d^4x\sqrt{|g|}\mathcal{L}_I,
  \label{eqn:general_action}
\end{equation}
where $\mathcal{L}_I$ is the {\em Lagrangian density}. We work in natural units $\hbar=c=1$ and set the reduced Planck mass $m_\mathrm{p} = {(8\pi G)}^{-1/2} = 1$. Dots denote differentiation with respect to cosmic time $\dot{f}\equiv \frac{d}{dt}f$, and primes denote differentiation with respect to conformal time $\prm{f}\equiv\frac{d}{d\eta}f$.

We begin by briefly summarising the classical theory of cosmological perturbations for a general scalar field, before discussing the quantisation of such a theory


\subsection{The classical action}
\label{sec:inflation}
Consider~\cite{Baumann+2009} a canonical scalar field $\phi$ minimally coupled to gravity $S= S_G + S_\phi$ with:
\begin{equation}
  \mathcal{L}_G = \frac{1}{2}R, 
  \qquad
  \mathcal{L}_\phi = \frac{1}{2}g^{\mu\nu}\nabla_\mu\phi\nabla_\nu\phi - V(\phi),
  \label{eqn:action}
\end{equation}
Extremising this action with respect to the fields $\phi$ and $g_{\mu\nu}$ recovers the Klein-Gordan and Einstein equations respectively:
\begin{align}
  \left( g^{\mu\nu}\nabla_\mu\nabla_\nu + \frac{dV}{d\phi} \right) \phi &= 0,
  \label{eqn:klein_gordon}\\
  G_{\mu\nu}\equiv R_{\mu\nu}-\frac{1}{2}g_{\mu\nu}R&= T_{\mu\nu},
  \label{eqn:einstein}
\end{align}
where the stress-energy tensor is:
\begin{equation}
  T_{\mu\nu} = \nabla_\mu\phi \nabla_\nu\phi - \frac{1}{2}g_{\mu\nu} \nabla_\alpha\phi \nabla^\alpha\phi +g_{\mu\nu} V(\phi).
  \label{eqn:SET}
\end{equation}

In cosmology, we assume that at zeroth order both the metric $g_{\mu\nu}$ and scalar field $\phi$ are homogeneous and isotropic. Applying these assumptions to equations~(\ref{eqn:klein_gordon})~\&~(\ref{eqn:einstein}), we find:
\begin{align}
  \dot{H}+H^2 &= -\frac{1}{3}\left( \dot{\phi}^2 - V(\phi) \right),
  \label{eqn:Raychaudhuri}\\
  0&=\ddot{\phi} + 3H\dot{\phi} + \frac{dV}{d\phi},
\end{align}
where the Hubble parameter $H = \dot{a}/a$.
