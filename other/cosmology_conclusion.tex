\chapter*{Part~\ref{part:cosmology} Conclusion}
\addcontentsline{toc}{chapter}{Part~\ref{part:cosmology} Conclusion}
\markboth{Part~\ref{part:cosmology} Conclusion}{Part~\ref{part:cosmology} Conclusion}

This part began by showing in Chapter~\ref{chap:kd} that almost all classical inflationary solutions begin in a generic kinetically dominated phase. However, as this statement is only valid within a semi-classical approximation, this phase can only be said to exist if it occurs after the Planck time.
Whether or not this phase occurs in a post-Planckian era can only be established by observation. If inflation was sufficiently short, then this pre-inflationary epoch may be observable as a suppression in power at low-$\ell$ in the $C_\ell$ spectrum, or in low-$k$ in the $\mathcal{P}_\mathcal{R}(k)$ spectrum. 

In Chapter~\ref{chap:rec} I showed that if one reconstructs the primordial power spectrum $\mathcal{P}_\mathcal{R}(k)$ from a Bayesian perspective there is some weak evidence for a suppression of power at low-$k$, as well as an anomaly at $\ell\sim30$. Whilst this by no means provides evidence for an observable kinetically dominated epoch, it does suggest the possibility that with better data there could be.

In order to gain more theoretical guidance on the precise predictions which the kinetically dominated universe makes about the primordial power spectrum, we have to gain a greater understanding of the quantum mechanics of this phase. The details of this are non-trivial, since as soon as one migrates away from a de-Sitter limit, the theory as to how to set initial conditions becomes far more murky. Chapter~\ref{chap:qv} has enumerated some of these issues, and provided an alternative possibility for quantising a kinetically dominated universe.

\section*{Future work}
\subsection{Generalising the kinetically dominated universe}


\subsection{Constraining the kinetically dominated universe}

The next task would be to ask if the data can provide any further insight into the quantum vacuum of the kinetically dominated universe. It would be particularly interesting to find out if the data themselves were capable of distinguishing between vacua. This could be done for the current set of cosmological data, or one could ask about the feasibility of future data sets in providing constraints on this portion of the unierse.

A full analysis would involve numerically integrating the quantum mechanical equations through the pre-inflationary phase all the way to horizon exit. Since these equations are highly oscillatory with time-varying coefficients, this would require a numerical method capable of tackling these. In fact, I have begun work on such a technique, which is detailed in Chapter~\ref{chap:RK}.

If one of these vacua is the ``right'' one, it still remains to be determined when, if at all, it was in its vacuum state. It may be that we can provide observational constraints on the precise value of this moment.

\subsection{Further constraints on inflation}
In addition, as a member of Planck core team II, I intend to continue more traditional observational reconstructions of inflationary and cosmological functions. 

First, it would be nice if one could take the analysis a further step backwards, and reconstruct the inflationary potential in a Bayesian manner. This would then predict a primordial power spectrum, which would subsequently be constrained by the Planck likelihoods.

Second, it would also be informative to take the reconstruction to the other end of the cosmic scale, and see what this Bayesian reconstruction method has to say about the $C_\ell$'s. In particular, it would be informative to find out whether the anomaly at $\ell\sim30$ is ``there'' in a Bayesian sense, or if any other less obvious anomalies are present.

\cleardoublepage{}
