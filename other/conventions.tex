\chapter*{Conventions}
\label{sec:cos:conventions}


\begin{itemize}
  \item I make full use of symbolic overloading; using the same symbol in different contexts where their mathematical meaning is distinct, but their physical meaning is related. For example:
    \begin{itemize}
      \item $f=f(x)$, where $f$ on the right hand side refers to a function ${f:X\to Y}$, but on the left hand side $f\in Y$ refers to the image of $x\in X$ under the function $f$.
      \item $\rho = \rho + \delta\rho$, where on the right hand side $\rho$ refers to the unperturbed solution, whilst on the left $\rho$ refers to the perturbed solution.
      \item $v_i = \partial_i v + v_i$, where on the left $v_i$ refers to a generic vector, but on the right $v$ refers to the helicity scalar part of the vector and $v_i$ refers to the helicity vector part.
    \end{itemize}
  \item I work using a metric with a positive signature $(+,-,-,-)$.
  \item Fourier transforms are defined so that Fourier synthesis carries the factors of $2\pi$:
    \begin{equation}
      f(\vk) = \int_{-\infty}^{\infty} d^3\vx\: e^{-i \vk\cdot\vx} f(\vx) \qquad f(\vx) = \int_{-\infty}^{\infty} \frac{d^3\vk}{{(2\pi)}^3}\: e^{i \vx\cdot\vk} f(\vk).
      \label{eqn:cos:fourier_transform}
    \end{equation}
  \item I work in natural units so that 
    \begin{equation}
      c = \hbar = G = k_B = 1,
      \label{eqn:cos:natural_units}
    \end{equation}
    but retain the reduced Planck mass for clarity:
    \begin{equation}
      \m = \frac{1}{8\pi G} = \frac{1}{8\pi}.
      \label{eqn:cos:reduced_planck_mass}
    \end{equation}
\end{itemize}

\cleardoublepage{}
