\chapter*{Conclusion: Part~\ref{part:statistics}}
\addcontentsline{toc}{chapter}{Conclusion: Part~\ref{part:statistics}}
\markboth{Conclusion: Part~\ref{part:statistics}}{Conclusion: Part~\ref{part:statistics}}


In this part I have given an account of the methods I have developed in order to further investigate the pre-inflationary universe. In Chapter~\ref{chap:pc} I detailed the construction of the next generation of nested sampling algorithms: \PolyChord{}. In Chapter~\ref{chap:RK} I described a novel method for solving differential equations with oscillating solutions: RKWKB\@.

\section*{Future}
\subsection*{Bayesian methods \& nested sampling}
The nature of nested sampling means that \PolyChord{} is applicable to a much broader range of astrophysical problems, far beyond its use in cosmology for this thesis. 

Alongside the content of this thesis, I have been working with other members of the department aiming to test Einstein's general theory of relativity in pulsar timing arrays, and with the AMI team to identify galaxies in clusters~\citep{Rumsey}. Further afield, I have been asked to work with teams in UCL to analyse FERMI x-rays in the search for dark matter and the Colorado DARE team to analyse the cosmic dark ages using radio wave data. The fast adoption of \PolyChord{} by the community is due to the fact that it is the first nested sampling algorithm capable of navigating complicated, high dimensional problems.

As well as this collaborative work, I am also excited about the scientific possibilities that \PolyChord{} has opened up outside of the field of astrophysics. The kind of problems it attacks are ubiquitous in science, and tend to stand out as long-standing unsolved theoretical issues. Examples range from the training of neural networks in machine learning to computing protein folds in biochemistry.

I have already developed a preliminary \PolyChord{} 2.0, capable of tackling even harder problems, and I anticipate the full exploration of the scientific applications forming a significant part of my future research. 

In a more theoretical Bayesian setting, it is my belief that \PolyChord{} is in fact merely the first in a large class of nested sampling algorithms, the true potential of which has only just begun to be tapped.

\subsection*{RKWKB}
The RKWKB approach has opened up an equally rich seam of potential research. There are many extensions which require immediate exploration. 

First, the multi-dimensional case is of interest. If this method could be extended to multiple coupled linear differential equations with oscillatory solutions, then the applications to cosmology are immediate. Currently, cosmological inference is dominated by the cost of computing transfer functions. The calculation of these involves the integration of coupled linear differential equations with time varying coefficients derived from the background cosmology. Implementing RKWKB in these contexts has the potential to lead to a new class of Boltzmann codes which remove the need for supercomputers in perform cosmological inference.

Second, the method could be improved if it no longer needed ``RK'' phases. If the same approach could be generalised so that the WKB stepping procedure was equally powerful regardless of the size of the oscillating term, then the method would be much cleaner and potentially more efficient.

Third, RKWKB has only thus far been applied to the linear case. Similar approaches by~\cite{Iserles03onthe} have been generalised to the non-linear case, and there is little reason to suggest that a similar approach would not also apply here. Along similar lines, it would be interesting to see if RKWKB can also be applied in a Lie group context as in~\cite{Iserles00lie-groupmethods}.



\cleardoublepage{}
