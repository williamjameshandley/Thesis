\section{Uniqueness theorem}
\label{sec:uniqueness_theorem}
We shall now prove that the solutions to the initial value problem of the master equation~\eqref{eqn:master_eq}:
\begin{align}
  \frac{\d{y}}{\d{\psi}}
  &=
  \sqrt{1-e^{-2y}} - \frac{\d{}}{\d{\psi}}\log \sqrt V,
  \label{eqn:IVP1}
  \\
  y(\psiz)
  &=
  y_0>0
  \label{eqn:IVP2},
\end{align}
are unique within any finite interval \(\psi\in[\psiz,\psi_1]\); i.e, if two positive solutions intersect at a point, then they intersect
everywhere.

We begin by putting a lower bound on \(y\) in the interval \([\psiz,\psi_1]\): From assumption~\eqref{eqn:conditions}, we know \(\dot{\phi}^2>\vellim^2>0\). If we unpack the definition of \(y\) using equations~\eqref{eqn:y_def},~\eqref{eqn:Ntrans},~\eqref{eqn:tau_def} and~\eqref{eqn:Friedmann_c}, one finds:
\begin{equation}
  y 
  = 
  \frac{1}{2}\log
  \left(\frac{\frac{1}{2}\dot{\phi}^2 + V(\phi)}{V(\phi)}\right) 
  > 
  \frac{\vellim^2}{4\Vm},
\end{equation}
where \(\Vm\) is the maximal value of \(V(\psi)\) in the interval \([\psiz,\psi_1]\). With this in hand we may prove the uniqueness of solutions of the initial value problem~\eqref{eqn:IVP1},~\eqref{eqn:IVP2} using standard techniques. For a good reference of such techniques the reader should consult the text by \citet{agarwal_1993}. In this case, we shall prove it using {\em Peano iteration}.


If one assumes that \(y(\psi)\) and \(z(\psi)\) are two distinct solutions, then their difference satisfies:
\begin{equation}
  \frac{\d{}}{\d{\psi}}(y-z)
  =
  \sqrt{1-e^{-2y}} - \sqrt{1-e^{-2z}}.
  \label{eqn:master_diff}
\end{equation}
If in addition one assumes they meet at a common point \(\psi_0\), so that \(y(\psi_0)=z(\psi_0)\), then integrating away from this position yields:
\begin{align}
  \abs{y(\psi)-z(\psi)}
  =&
  \abs{\int_{\psi_0}^\psi \sqrt{1-e^{-2y}}
  - \sqrt{1-e^{-2z}}\:\:d\psi}
  \nonumber\\
  &
  \le \int_{\psi_0}^\psi \abs{\sqrt{1-e^{-2y}}
  - \sqrt{1-e^{-2z}}}d\psi.
  \label{eqn:ineq_1}
\end{align}
A generic property of the function \(f(y)=\sqrt{1-e^{-2y}}\) is that in the interval \(\left[\phantom(\frac{\vellim^2}{4\Vm},\infty\right)\phantom]\) it is {\em Lipschitz continuous\/}:
\begin{equation}
  \abs{\sqrt{1-e^{-2y}} - \sqrt{1-e^{-2z}}} \le L\abs{y-z},
  \label{eqn:Lipschitz}
\end{equation}
where \(L\) is the {\em Lipschitz constant}, taking the value:
\begin{equation}
  L
  = 
  \frac{\exp{\left(-\frac{\vellim^2}{2\Vm}\right)}}
  {\sqrt{1-\exp{\left(-\frac{\vellim^2}{2\Vm}\right)}}} 
  > 0.
\end{equation}
Applying Lipschitz continuity~\eqref{eqn:Lipschitz} to the inequality in~\eqref{eqn:ineq_1} gives:
\begin{equation}
  \abs{y(\psi)-z(\psi)} 
  \le
  L\int_{\psi_0}^\psi\abs{y(\psi)-z(\psi)}\d{\psi}.
  \label{eqn:ineq_2}
\end{equation}
Further, if the maximum value of the difference of \(|y-z|\) between \(\psi_0\) and \(\psi\) is \(\Delta\), then the above implies:
\begin{equation}
  \abs{y(\psi)-z(\psi)} 
  \le
  L\Delta\abs{\int_{\psi_0}^\psi \d{\psi}}= 
  L\Delta\abs{\psi-\psi_0}.
\end{equation}
Applying this inequality back into~\eqref{eqn:ineq_2} shows:
\begin{equation}
  \abs{y(\psi)-z(\psi)} 
  \le
  L^2\Delta\int_{\psi_0}^\psi\abs{\psi-\psi_0}\d{\psi} =
  L^2\Delta\frac{\abs{\psi-\psi_0}^2}{2!}.
\end{equation}
Applying this back into~\eqref{eqn:ineq_2} yields:
\begin{equation}
  \abs{y(\psi)-z(\psi)} 
  \le
  L^3\Delta\frac{\abs{\psi-\psi_0}^3}{3!},
\end{equation}
and by induction on \(n\in\mathbb{N}\) we find that:
\begin{equation}
  \abs{y(\psi)-z(\psi)} 
  \le
  L^n\Delta\frac{\abs{\psi-\psi_0}^n}{n!}.
\end{equation}
As \(n\to\infty\) the term on the right-hand side drops to \(0\), and therefore \(|y(\psi)-z(\psi)|=0\). Thus, if \(y\) and \(z\) are equal at some point \(\psi_0\), then they are equal at all points \(\psi\) within any finite interval \([\psiz,\psi_1]\). Two separate solutions cannot ``cross over'', and if one positive solution \(f\) is initially less than a second solution \(h\) at \(\psiz\), \(f(\psiz)<h(\psiz)\), then \(f(\psi_1)<h(\psi_1)\) for any finite \(\psi_1\).  


