\section{Runge-Kutta-Fehlberg}
\label{sec:rkf}

A general explicit RK method can be written as:
\begin{align}
  y_{n+1} &= y_n + h\sum\limits_{i=1}^{s} b_i k_i, \label{eqn:rk_step} \\
  k_s &= f(t_n + c_s h, y_n + h \sum\limits_{i=1}^{s-1}a_{si} k_i),
\end{align}
where the coefficients \(\{c_i,a_{si}\}\) are determined by the choice of method and are typically written in a Butcher tableau (Table~\ref{tab:RKexplicit}). 

A particularly efficient example is the Runge-Kutta-Fehlberg \(4(5)\) method which uses an embedded approach. It performs a fourth order step and a fifth order step, and uses the difference between these as an estimate of the error. Impressively, both steps are calculated using the same values of \(\{k_i\}\) (but different values of \(\{b_i\}\)), and hence the method only requires five function evaluations of \(f\) per step. Its Butcher tableau is detailed in Table~\ref{tab:rkf45}.

\begin{table}[tp]
  \centering
  \begin{equation*}
  \begin{array}{l | c c c c c}
    0      \quad &               &              &              &         &   \\
    c_2    \quad & \quad a_{21}  &              &              &         &   \\
    c_3    \quad & \quad a_{31}  & \quad a_{32} &              &         &   \\
    \vdots \quad & \quad \vdots  & \quad \vdots & \quad \ddots &         &   \\
    c_s    \quad & \quad a_{s1}  & \quad a_{s2} & \quad \cdots & \quad a_{s,s-1} & \\ \midrule
    & \quad b_{1}   & \quad b_{2}  & \quad \cdots & \quad b_{s-1}  & \quad b_{s}
  \end{array}
\end{equation*}

  \caption{Butcher tableau for a general explicit RK method.\label{tab:RKexplicit}
}
\end{table}

\begin{table}[tp]
  \centering
    \begin{equation*}
    \begin{array}{l | c c c c c c}
      0      &           &            &             &             &        &\\
      \frac{1}{4}    & \frac{1}{4}       &            &             &             &        &\\
      \frac{3}{8}    & \frac{3}{32}      & \frac{9}{32}       &             &             &        &\\
      \frac{12}{13}  & \frac{1932}{2197} & -\frac{7200}{2197} & \frac{7296}{2197}   &             &        &\\
      1      & \frac{439}{216}   & -8         & \frac{3680}{513}    & -\frac{845}{4104}   &        &\\
      \frac{1}{2}    & -\frac{8}{27}     & 2          & -\frac{3544}{2565}  & \frac{1859}{4104}   & -\frac{11}{40} &\\\midrule
      & \frac{25}{216}    & 0          & \frac{1408}{2565}   & \frac{2197}{4104}   & -\frac{1}{5}   & 0      \\
      & \frac{16}{135}    & 0          & \frac{6656}{12825}  & \frac{28561}{56430} & -\frac{9}{50}  & \frac{2}{55}
    \end{array}
  \end{equation*}

  \caption{Butcher tableau for the embedded Runge-Kutta-Fehlberg \(4(5)\) method.\label{tab:rkf45}                                                             }
\end{table}
