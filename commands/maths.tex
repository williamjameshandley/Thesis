

% Math operators
\DeclareMathOperator{\sech}{sech}
\DeclareMathOperator{\csch}{csch}
\DeclareMathOperator{\arcsec}{arcsec}
\DeclareMathOperator{\arccot}{arccot}
\DeclareMathOperator{\arccsc}{arccsc}
\DeclareMathOperator{\arccosh}{arccosh}
\DeclareMathOperator{\arcsinh}{arcsinh}
\DeclareMathOperator{\arctanh}{arctanh}
\DeclareMathOperator{\arcsech}{arcsech}
\DeclareMathOperator{\arccsch}{arcCsch}
\DeclareMathOperator{\arccoth}{arcCoth} 



% prime
\newcommand{\prm}[1]{{{#1}^\prime}} 
% double prime
\newcommand{\dprm}[1]{{{#1}^{\prime\prime}}}

% star
\newcommand{\str}[1]{{#1}^\ast}


%  mean 
\newcommand{\mean}[1]{{\left\langle{#1}\right\rangle}}
%  'of order'
\newcommand{\bigO}[1]{{\sim\mathcal{O}{\left(#1\right)}}}


\newcommand{\ket}[1]{\left|#1\right\rangle}
\newcommand{\bra}[1]{\left\langle#1\right|}
\newcommand{\braket}[2]{\left\langle#1\middle|#2\right\rangle}
\newcommand{\bracket}[3]{\left\langle#1\middle|#2\middle|#3\right\rangle}

% a text fraction
\def\tfrac#1#2{{\textstyle\frac{#1}{#2}}}


% Differentiation
\newcommand{\difrac}[2]{\frac{d #1}{d #2}}    % shortcut for dx/dy

% Absolute value function
\newcommand{\abs}[1]{\left|#1\right|}  


