% Corrections for PhD thesis of Will Handley

\documentclass[11pt]{article}
%\pagestyle{empty}
\usepackage{times}
\usepackage{amsmath}
%\usepackage{mathpsb}
\topmargin 0.1in
\oddsidemargin 0.0in
\evensidemargin -0.1in
\textheight 9.0in
\textwidth 6.25in
\headheight 0in
\headsep 0in
\parindent 0in
\parskip \bigskipamount

%\oddsidemargin 0.2cm
%\evensidemargin 0.2cm
%\textwidth 16.05cm
%\textheight 23.5cm
%\headheight -0.5cm
%\headsep 0.cm
%\topmargin -0.2cm
%\parskip \baselineskip
%\parindent 0.0cm

\usepackage{xcolor}
\usepackage{listings}
\lstset{basicstyle=\ttfamily,
  showstringspaces=false,
  commentstyle=\color{red},
  keywordstyle=\color{blue}
}

\newcommand{\todo}[1]{{\color{red} #1}}


\begin{document}

\noindent{\large{\textbf{Corrections for the Ph.D.\ thesis
of William Handley (Department of Physics, University of Cambridge)
entitled ``Kinetic initial conditions for inflation: theory,
observation and methods''}}}

\paragraph{General}
\begin{enumerate}
\item No need to have a colon in the text before every equation.
%  \begin{itemize}
%    \item Fixed with the command:
%      \begin{lstlisting}[language=bash]
%        find appendices chapters -type f -name "*.tex" -exec \
% sed -i 's/:\s*$//' {} \;
%      \end{lstlisting}
%  \end{itemize}
%  Note: this may be quickly version-controlled back, if this is not wha you meant
\end{enumerate}

\paragraph{Chapter 1}
\begin{enumerate}
\item Page 2, Fig. 1.2. State what frequency this is at.
  \begin{itemize}
    \item Done (143GHz).
  \end{itemize}
\item Page 3, first paragraph. Reference Fig. 1.2 rather than 1.1 for
  the ``signal generated by the beginning of the universe''.
  \begin{itemize}
    \item Fixed.
  \end{itemize}
\item Page 3, third paragraph. Missing full-stop at end of paragraph.
  \begin{itemize}
    \item Fixed.
  \end{itemize}
\end{enumerate}

\paragraph{Chapter 2}
\begin{enumerate}
\item Page 8, before Eq. (2.7). Add a historical note that Einstein
  would reverse this argument about conservation of stress-energy.
  \begin{itemize}
    \item Added a footnote:
    \item Historical note: Einstein in fact derived this argument in reverse. He defined the Einstein tensor (2.6) by adjusting the Ricci tensor so that the stress energy tensor is conserved by construction. In the more modern approach presented here, the definition of the Einstein tensor arises naturally  from variational principles.
  \end{itemize}
\item Page 8, end of Sec. 2.3. Einstein equations 2.4 $\rightarrow$
  Einstein equations (2.4).
  \begin{itemize}
    \item Corrected, and all other instances throughout the text also found (with a grep)
  \end{itemize}
\item Page 10, after Eq. (2.16). You cannot take $a(t_0)=1$ if you
  take $k=\pm 1$ for closed and open models, as you state in
  Eq. (2.11).
  \begin{itemize}
    \item Adjusted Table 2.1 and Eq. (2.11) so that $k$ may take any real value. (i.e. $k=\pm 1,0$ $\rightarrow$ $k>,=,<0$)
  \end{itemize}
\item Page 11, before Eq. (2.19). There are some errors in this
  derivation. It is true that $p_\chi$ is conserved, but raising the
  index then gives $a^2 p^\chi = \text{const.}$. Moreover, $p^\chi
  \propto 1/(\lambda a)$, which finally gives $\lambda / a =
  \text{const.}$. 
  \begin{itemize}
    \item Corrected so that the argument now reads:
    \item Using the metric to raise the indices, one finds that \(p^\chi(t_2){a(t_2)}^2=p^\chi(t_1){a(t_1)}^2\). Identifying the physical momentum \({a p^\chi \propto k^r \propto \lambda^{-1}}\) where \(\lambda\) is the wavelength of the photon, one finds\ldots 
  \end{itemize}
\item Page 12, start of Sec. 2.4.2. It \textbf{is} convenient.
  \begin{itemize}
    \item Corrected
  \end{itemize}
\item Page 14, after Eq. (2.27). The final sentence of this paragraph
  is odd. The curvature of space, as measured by the Ricci scalar of
  the 3D sections, always decreases in an expanding universe (as
  $k/a^2$). What you mean is that for ordinary matter, the energy
  density of the matter falls more rapidly so that curvature comes to
  dominate the dynamics.
  \begin{itemize}
    \item Corrected thus:
    \item This is natural, as the effect of spatial curvature on the energy and dynamics of the universe has a slower decay rate in comparison with ordinary matter (equation 2.16).
  \end{itemize}
\item Page 15, Eq. (2.28). Rewrite this as 
%
\begin{equation}
\frac{\ddot{a}}{a} = - (1+3w) \rho \, . \tag{2.28}
\end{equation}
%
\begin{itemize}
  \item Done.
\end{itemize}
\todo{
\item Page 17, first sentence. Why does spatial uniformity have
  anything to do with the flatness problem?
}
\item Page 17, after Eq. (2.34). This is not the ``equation of a
  harmonic oscillator'' unless $V(\phi) \propto \phi^2$.
  \begin{itemize}
    \item Change ``harmonic oscillator'' to ``particle''
  \end{itemize}
  \todo{
\item Page 19, Table 2.3. Is calling $E_{ij}$ the spatial shear tensor
  conventional. It is $\dot{E}_{ij}$ that determines the shear of the
  worldlines of constant coordinate position.
}
\item Page 20, after Eq. (2.41). For more detail, see \textbf{Baumann
    (2009, p.~48)}. Also, you might want to add a note here that
  scalar field fluctuations are generally non-adiabatic, i.e., they do
  not satisfy Eq. (2.41).
  \begin{itemize}
    \item citep changed to citet
    \item Added comment ``, although it is not in general satisfied
      by the pressure perturbations generated by scalar fields''
  \end{itemize}
\item Page 20, Eq. (2.44). You need the $\delta$ applied to the whole
  equation, not just the stress-energy tensor, i.e., you need to
  account for the fluctuations in the connection for the covariant
  derivative.
  \begin{itemize}
    \item Corrected by moving the $\delta$ outside the derivative.
  \end{itemize}
\item Page 20, Eq. (2.46). This should be $\varphi(t,\mathbf{k})$ in
  the integrand.
  \begin{itemize}
    \item Corrected
  \end{itemize}
  \todo{
\item Page 22, Eq. (2.52). The conventions here seem inconsistent with
  your earlier Eq. (2.39), which would give $T^i{}_0 \propto \delta
  q^i$, whereas your Eq. (2.52) assumes $T^0{}_i \propto \delta
  q_i$. Please check and revise if necessary. This may also propagate
  to Eq. (2.54). You also need to check whether you are consistently
  assuming that $E_{ij}$ is trace-free, since on expects a Laplacian
  of $E$ in Eq. (2.54) if $E_{ij}$ is indeed trace-free.
}
\item Page 23, Table 2.4. The comoving orthogonal gauge has $\delta q
  = 0$ and $B=0$, not $E=0$. Is this what you intend here?
  \begin{itemize}
    \item Baumann's Tasi lectures (pp 136-9) name this choice as ``comoving'' due to the vanishing of the scalar momentum density. I have now added ``comoving orthogonal'' to the list
  \end{itemize}
\item Page 23, Eq. (2.55). Should the $\mathcal{R}$ on the right-hand
  side be $\Psi$. Also, include factor of $m_{\text{pl}}^2$ in this
  term to be consistent with earlier claim at bottom of Page
  xiii. Finally, note that this assumes no anisotropic stress.
  \begin{itemize}
    \item The term on the right is actually $\mathcal{R}$ (I derived this expression myself using maple). 
    \item The term of $m_{\text{pl}}^2$ has been added -- well spotted (I generally set it to $1$ for the purposes of computer algebra). I have also added it to the $\rho,P\sim m_\text{pl}^2 H^2$ expression beneath it.
    \item a note has been added about the assumption of no anisotropic stress
  \end{itemize}
\item Page 23, Eq. (2.57). $\delta_i \rightarrow \partial_i$ in
  integrand.
  \begin{itemize}
    \item Corrected
  \end{itemize}
\item Page 24, before Eq. (2.65). Explain how you get $z''/z = a''/a$
  for de Sitter space (where $z = a\dot{\varphi}/H$ is problematic in
  the limit of exact de Sitter space).
  \begin{itemize}
    \item Adjusted the text with this explanation
    \item As discussed in Section 2.7.2, pure de Sitter space may be achieved in an inflationary scenario if \(H,\phi=\mathrm{const}\), and is approximated under the conditions of slow roll~(2.36).  Thus, in a space where \(H\) and \(\dot\phi\) are both constant, \({z^{\prime\prime}/z = a^{\prime\prime}/a = 2{(\eta_\mathrm{end}-\eta)}^{-2}}\). Taking the limit as \(\dot{\phi}\to0\) creates a pure de Sitter scenario, and the differential equation~(2.62) has the general solution:
  \end{itemize}
  \todo{
\item Page 24, after Eq. (2.66). Make Bunch-Davies $A_k$ and $B_k$
  here consistent with choice in Chapter 6.
}
\todo{
\item Page 24, Eq. (2.68). The power spectrum here is the dimensional
  power spectrum, universally denoted by $P_{\varphi}(k)$. Follow
  these conventions, reserving $\mathcal{P}_\varphi(k)$ for the
  dimensionless spectrum [also sometimes denoted
  $\Delta_\varphi^2(k)$]. You need to check throughout that you are
  being consistent in your usage since. For example, later in
  Eqs. (3.131) and (3.133) these are both the dimensional power
  spectrum, but are denoted by $\mathcal{P}$ and $P$, respectively.
}
\item Page 25, Eq. (2.72). Add full-stop at end of equation.
  \begin{itemize}
    \item Corrected
  \end{itemize}
  \todo{
\item Page 25, after Eq. (2.75). In exact de Sitter, $\dot{\phi} = 0$,
  so have to be careful here.
}
\todo{
\item Page 26, Eq. (2.78). Define $\mathbf{p}$. Also, what is meant by
  ``the temperature field is anisotropic on account of the
  relativistic nature of photons''?
}
\item Page 26, Eq. (2.80). Missing complex conjugate on $Y_{\ell m}$.
  \begin{itemize}
    \item Corrected
  \end{itemize}
\item Page 26, Eq. (2.81). Missing complex conjugate on one of the
  $a_{\ell m}$.
  \begin{itemize}
    \item Corrected
  \end{itemize}
\item Page 27, first sentence. This does not make sense.
  \begin{itemize}
    \item Reordered to:
    \item where \(\Delta_{X\ell}(k)\) are transfer functions. These transfer functions \ldots  sky we see today. 
 \({X\in\left\{ T,E,B \right\}}\) labels the components of the photon field corresponding to the temperature anisotropy, and the \(E\) and \(B\) polarisation modes respectively.
  \end{itemize}
\end{enumerate}

\paragraph{Chapter 3}
\begin{enumerate}
    \todo{
\item Need to revise the proof of kinetic dominance being generic
  accounting for the error in Eq. (3.33) and the fact that $y\geq 0$.
}
\item Page 37, Eq. (3.44). Missing full-stop at end of equation.
  \begin{itemize}
    \item Corrected
  \end{itemize}
\item Page 40, Eq. (3.62). Make the brackets in ${}_2 F_1$ larger.
  \begin{itemize}
    \item Corrected
    \item also Eq. (3.63)
  \end{itemize}
\item Page 44, after Eq. (3.96). Write the equation numbers 3.69 and
  3.73 in brackets.
  \begin{itemize}
    \item Corrected
  \end{itemize}
\item Page 44, Eq. (3.97). Comma rather than full-stop at end of
  equation.
  \begin{itemize}
    \item Corrected
  \end{itemize}
\item Page 44, Eq. (3.101). Missing full-stop at end of equation.
  \begin{itemize}
    \item Corrected
  \end{itemize}
\item Page 45, Eq. (3.103). Missing full-stop at end of equation.
  \begin{itemize}
    \item Corrected
  \end{itemize}
\item Page 46, Eq. (3.104). Remove the $\sum_i \rho_i$ since you are
  discussing the case where the ```universe is spatially flat and
  contains only the inflaton field''.
  \begin{itemize}
    \item Removed
  \end{itemize}
\item Page 46, after Eq. (3.108). The number of $e$-folds $N_\ast$
  between the pivot scale $k_\ast$ \textbf{exiting the Hubble radius}
  and the end of inflation \ldots
  \begin{itemize}
    \item Amended
  \end{itemize}
\item Page 47, Eqs. (3.110--3.112). Use the same shorter arrow as in
  Eqs. (3.102) and (3.103).
  \begin{itemize}
    \item Fixed
  \end{itemize}
\item Page 51, Eq. (3.126). Remove equation number from this equation,
  keeping the number for the second in this displayed group. Also,
  should the $\phi$ be a $\psi$ in the derivative on the left-hand
  side (and in Eqs. 3.128 and 3.129 too)?
  \begin{itemize}
    \item Numbering fixed
    \item Yes, it should be $\psi$
  \end{itemize}
\item Page 52, Eq. (3.129). Add full-stop at end of sentence.
  \begin{itemize}
    \item Fixed
  \end{itemize}
\item Page 53, Fig. 3.8. $f(\phi) \rightarrow f(\psi)$ in caption.
  \begin{itemize}
    \item Fixed
  \end{itemize}
  \todo{
\item Page 54, around Eq. (3.131). Add discussion that this is an
  extrapolation of the slow-roll result and only approximate for any
  fast-roll inflationary phase. Ideally, add a figure comparing the
  approximate result to that from integrating the mode equations directly.
}
\item Page 55, Fig. 3.10. State in the caption the \emph{total} number
  of $e$-folds of inflation assumed here.
  \begin{itemize}
    \item Stated now to be $N_\mathrm{tot}=65$ e-folds of inflation
  \end{itemize}
\item Page 57, first paragraph. Add a reference to Chapter 5 here.
  \begin{itemize}
    \item Added:
    \item , updated results for which are reported in Chapter~5.
  \end{itemize}
\item Page 57, Eq. (3.133). Add full-stop at end of equation (and
  watch out for power spectrum definitions).
  \begin{itemize}
    \item Fixed
  \end{itemize}
\item Page 57, after Eq. (3.133). \ldots yields the standard matter
  \textbf{and} CMB power spectra \ldots
  \begin{itemize}
    \item Fixed
  \end{itemize}
\item Page 57, Eq. (3.134). Missing a square on left-hand side. After
  this equation, define BVS at first use.
  \begin{itemize}
    \item fixed missing square 
    \item changed BVS to Boyanovsky et al.
  \end{itemize}
\item Page 61, final equation. Add an equation number (you seem to be
  numbering all equations, which is fine) and a full-stop at the end
  of the equation.
  \begin{itemize}
    \item Fixed
  \end{itemize}
\item Page 64, second paragraph. \ldots conditions to spatially-flat
  polynomial and exponential inflation models.
  \begin{itemize}
    \item Corrected.
  \end{itemize}
\end{enumerate}

\paragraph{Chapter 4}
\begin{enumerate}
\item Page 68, after Eq. (4.11). $N_\ast \sim 50$\textbf{--60} (i.e.,
  remove spurious spaces) and similarly right at the end of Sec. 4.1.
  \begin{itemize}
    \item Both fixed
  \end{itemize}
\item Page 69, first paragraph. Delete spurious ``make'' from ``one
  may make place''.
  \begin{itemize}
    \item Corrected
  \end{itemize}
\item Page 69, bullet-point D. Add full-stop at end.
  \begin{itemize}
    \item Corrected
  \end{itemize}
\item Page 69, around Eq. (4.12). Replace ``Planck surface'' with
  Planck circle. Also, define the angle $\theta$.
  \begin{itemize}
      \item Done. $\theta$ defined by: \(\tan\theta  = \dot{\phi}/(m\phi)\).
  \end{itemize}
\item Page 70, Eq. (4.20). Add full-stop at end of equation.
  \begin{itemize}
    \item Corrected
  \end{itemize}
\item Page 70, Eq. (4.23). Give oscillatory terms also in asymptotic
  expression.
  \begin{itemize}
    \item Done. Introduced new constants $C$ and $D$ to absorb additional complexity, and had to split the equation onto two lines.
  \end{itemize}
\todo{
\item Page 71, penultimate paragraph. Reword this to make the argument
  more precise.
}
\end{enumerate}

\paragraph{Chapter 5}
\begin{enumerate}
\item Page 75. This is an example of where the ordering of material is
  not ideal. The reader first meets the Bayes factor and evidence
  here, but they are only properly introduced later in Chapter
  7. Perhaps just add a forward reference to that chapter.
  \begin{itemize}
      \item Adjusted the second paragraph of the introduction to read:
      \item In this chapter, I reconstruct the primordial power spectrum of curvature perturbations from a model-independent standpoint using Bayesian techniques applied to Planck 2015 data (\ldots). A more detailed discussion of Bayesian analysis and techniques is given in Chapters~7--9. It is worth remarking that the analysis detailed in this chapter only became possible after the creation of PolyChord, which is detailed in Chapter~9. 

  \end{itemize}
\item Page 76, Fig. 5.1. Be more explicit about how the power spectrum
  is extrapolated outside the end-points.
  \begin{itemize}
      \item Added some dashed lines to Figure 5.1, and in the caption the text:
      \item Outside this range the function is extrapolated smoothly as indicated by the dashed lines in the diagram
  \end{itemize}
\item Page 76, second paragraph. Add Chapter 9 to the references about
  POLYCHORD.
  \begin{itemize}
      \item Done
  \end{itemize}
\item Page 77, Fig. 5.2. $\ell = k D_{\text{rec}}$. Also, aren't
  $1\,\sigma$, $2\,\sigma$ and $3\,\sigma$ contours shown?
  \begin{itemize}
      \item Corrected, and yes they are. Adjusted accordingly.
  \end{itemize}
\item Page 78, final sentence. The reference to Sect. 4.4 seems to be
  leftover from the Planck inflation paper. Similarly the reference to
  Sect. 9 on Page 79.
  \begin{itemize}
      \item Added references to the Planck paper at these points
  \end{itemize}
\end{enumerate}

\paragraph{Chapter 6}
\begin{enumerate}
\item Page 82, first sentence. Replace ``delay'' with defer.
    \begin{itemize}
        \item Done
    \end{itemize}
\item Page 82, Eq. (6.2). Full-stop at end of sentence.
    \begin{itemize}
        \item Done
    \end{itemize}
\item Page 82, after Eq. (6.2). Klein-\textbf{Gordon}.
    \begin{itemize}
        \item Done
    \end{itemize}
\item Page 83, Eq. (6.9). The notation here for the metric
  perturbations is different to that in
  Chapter 2; why not use the same? Also, there is a missing $a^2$ in
  last line.
  \begin{itemize}
      \item Switched notation. Also restricted to just scalar perturbations, since vector ones are no longer used
  \end{itemize}
\item Page 83, Eqs. (6.11) and (6.13). The notation was $S^{(2)}$ in
  Chapter 2.
  \begin{itemize}
      \item Corrected subscript to superscript
  \end{itemize}
\item Page 83, last sentence. This does not make sense (``and that
  canonical commutation relation:'').
  \begin{itemize}
      \item Changed to ``along with the canonical commutator relation''
  \end{itemize}
\item Page 84, Eq. (6.23). Replace full-stop with comma at end of
  equation.
  \begin{itemize}
      \item Done
  \end{itemize}
\item Page 86, Eq. (6.28). Need full-stop at end of sentence.
    \begin{itemize}
        \item Done
    \end{itemize}
    \todo{%
\item Page 87, after Eq. (6.30). Tighten up the argument why $z$ is
  approximately proportional to $a$ during inflation.
  }
\item Page 87, Eq. (6.32). Clarify that the bi-tensor is only used in
  the limit so no need to distinguish argument of metric.
  \begin{itemize}
      \item Added a clause after the equation:
        \item ``where, as we are taking the coincidence limit, the metric \(g_{\mu\nu}\) may be evaluated at either \(x\) or \(\prm{x}\).''
  \end{itemize}
\item Page 88, Eq. (6.34). Add some (square) brackets to clarify what
  is the integrand.
  \begin{itemize}
      \item Done
  \end{itemize}
\item Page 88,  first sentence of Sec. 6.6.2. Remove spurious
  ``take''.
  \begin{itemize}
      \item Done
  \end{itemize}
\item Page 89, first paragraph of Sec. 6.7. Give some mention of
  Chapter 3 here, rather than just the reference to your 2014 paper. 
  \begin{itemize}
      \item Altered to ``As is detailed in Chapter 3 (and in Handley et al., 2014), the classical\ldots''
  \end{itemize}
\end{enumerate}

\paragraph{Conclusion: Cosmology}
\begin{enumerate}
\item Page 91, first paragraph. \ldots or \textbf{at} low-$k$ in the
  \ldots
  \begin{itemize}
      \item Done
  \end{itemize}
\item Page 92, first paragraph. Correct ``unierse''.
    \begin{itemize}
        \item Done
    \end{itemize}
\end{enumerate}

\paragraph{Chapter 7}
\begin{enumerate}
\item Page 98, Eq. (7.10). Should the $n$ be $N$ in this equation?
    \begin{itemize}
        \item Yes. Corrected
    \end{itemize}
\item Page 102, Eq. (7.24). Replace full-stop with comma at end of
  equation.
  \begin{itemize}
      \item Done
  \end{itemize}
\item Page 103, Eq. (7.29). Full-stop at end of equation.
    \begin{itemize}
        \item Done
    \end{itemize}
\item Page 107, Eq. (7.39). Full-stop at end of equation.
    \begin{itemize}
        \item Done
    \end{itemize}
\item Page 110, around Eq. (7.47). Odd sentence construction (multiple
  colons in sentence) above the equation. Remind the reader what is
  the relation of the $\theta_i$ to $\Theta$ (is the latter the set of
  the former?). 
  \begin{itemize}
      \item Yes, although in being consistent with my overloading strategy, $\Theta$ should really be $\theta$.
      \item Sentence now reads: ``In the general \(D\)-dimensional case \(f=f(\theta)=f(\theta_1,\ldots,\theta_D)\) one calculates \(D\) conditional distributions \(\{f_i:i=1,\ldots,D\}\), by marginalising over parameters with indices greater than \(i\) and conditioning on parameters with indices less than \(i\):''
      \item For this section (7.A.1), the final paragraph is in fact better suited to the beginning of the next section (7.A.2). I have moved this and adjusted accordingly.
  \end{itemize}
\item Page 110, Eq. (7.48). Should the lower limit of the integral be
  $-\infty$ (as in Eq. 7.46)?
  \begin{itemize}
      \item Yes. Corrected
  \end{itemize}
\item Page 111, after Eq. (7.56). Remove capitalisation on ``We find
  \ldots''.
  \begin{itemize}
      \item Done
  \end{itemize}
\item Page 112, before Eq. (7.59). Add round brackets around 7.50 in
  the text.
  \begin{itemize}
      \item Done
  \end{itemize}
\item Page 112, Eq. (7.63). Add full-stop to abbreviation ``const''.
    \begin{itemize}
        \item Done
    \end{itemize}
\item Page 114, first paragraph. Define $\tau$ and $L$.
    \begin{itemize}
        \item Reworded to:
        \item It is important to note that the chain \(C_T\) will comprise \(T\) samples which are {\em correlated}. Consider Figure~7.7 once more, where the shortest and longest lengthscales of the probability region are \(\sigma_{\min{}}\) and \(\sigma_{\max{}}\) respectively. If the proposal distribution amounts to a random walk with step size of order the shortest lengthscale \(\varepsilon\sim\sigma_{\min{}}\), then the diffusive Brownian nature of a random walk leads one to expect the timescale to fully traverse the region will be:
            \begin{equation}
                \tau \sim {\left( \frac{\sigma_{\max{}}}{\sigma_{\min{}}} \right)}^2.
                \label{eqn:sm:MH_timescale}
            \end{equation}
            It is this timescale which determines the degree of correlation between samples within a chain \(C_T\).

            In theory, given that the relation~\eqref{eqn:sm:MH_timescale} has no dimensionality attached to it, Metropolis Hastings can be extremely successful in high dimensions. However, there are several issues.

    \end{itemize}
\item Page 116, Eq. (7.71). Should this be $\mathcal{L}^\beta$ in the
  integrand in the denominator?
  \begin{itemize}
      \item Yes. Corrected
  \end{itemize}
\item Page 116, Eq. (7.72). Add integration measure $d\beta$.
    \begin{itemize}
        \item Done
    \end{itemize}
\end{enumerate}

\paragraph{Chapter 8}
\begin{enumerate}
\item Page 118, Eqs. (8.6) and (8.7). Explain that these only hold for
  a log-normally distributed $\mathcal{Z}$.
\item Page 119, Section 8.2. Improve the interfacing with Chapter 9 so
  that ``clusters'' do not appear out of the blue at this point.
\item Page 120, before Eq. (8.26).  \ldots (8.20) \textbf{one} finds:.
\item Page 121, Eq. (8.47). Spurious equation number. Also, need
  full-stop at end of last equation in this grouping.
\end{enumerate}

\paragraph{Chapter 9}
\begin{enumerate}
\item Page 128, second paragraph. 2--6 rather than 2---6.
\item Page 130, after Eq. (9.2). \ldots but exhibits a law of \ldots
\item Page 135, last sentence. Reference should be to Fig. 9.5 not
  9.6.
\item Page 140, Fig. 9.11. There is something odd going on with the
  fonts in CAMSPEC+commander.
\end{enumerate}

\paragraph{Chapter 10}
\begin{enumerate}
\item Page 146, after Eq. (10.11). Missing $\pm i$ in exponent of
  in-line equation.
\item Page 146, Eq. (10.13). Missing summation over $i$ in last term.
\item Page 147, Eq. (10.22). Add comma at end of equation.
\item Page 147, second paragraph. Explain clearly here why you perform
  the WKB steps for $x$ and $\dot{x}$ separately, and, ideally,
  justify why this helps over just propagating $x$.
\item Page 148, second paragraph. Need apostrophe in ``users''.
\end{enumerate}

\paragraph{Conclusion: Methods}
\begin{enumerate}
\item Page 153, second paragraph of Bayesian methods and nested
  sampling. Replace ``x-rays'' with X-rays.
\end{enumerate}



\paragraph{Further changes}
\begin{enumerate}
    \item Added a Lasenby 2009 reference to Chapter 6 (in light of PRD review process)
    \item Chapter 6 is now published in PRD: References have been updated accordingly in the abstract and introduction
\end{enumerate}

\end{document}































