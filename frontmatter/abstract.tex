\abstractnum{}   % Format the title like a chapter title

\begin{abstract}
%\thispagestyle{empty}
\noindent
This thesis is concerned with the initial conditions for inflation, and the construction of methods to aid in the observational and theoretical analysis of the early universe.
Chapter~\ref{chp:out} outlines the context and content of the thesis. After this, the thesis is divided into two parts.
The first chapters of each part (Chapters~\ref{chp:cos} and~\ref{chp:bay}) are introductory, establishing basic theory and notation. The remainder of the thesis is entirely my own work, except where references explicitly state otherwise.

Part~\ref{part:cosmology} contains theoretical and observational work in early-universe cosmology, and is divided into four chapters. 
Chapter~\ref{chp:cos} introduces the inflationary cosmological theory relevant to this thesis. 

Chapter~\ref{chp:kd} is published in~\cite{Handley+2014}, and proves the theoretical result that almost all classical universes begin at a finite time in the past in a generic kinetically dominated state. Classical kinetically dominated universes are examined in detail, and the possible observable consequences are postulated. 
Chapter~\ref{chp:kt} details further theoretical observations into the kinetically dominated universe that have arisen since the publication of~\cite{Handley+2014}. 

Chapter~\ref{chp:rec} was published as part of the~\cite{planck2015-a24}, and gives model-independent reconstructions of the primordial power spectrum of curvature perturbations. The results are consistent with the concordance $\Lambda$CDM cosmology, but show hints of possible effects of kinetic dominance.

Chapter~\ref{chp:qv} details theoretical work in the quantum mechanics of the early universe, and has been submitted for publication in Physical Review. A novel approach for defining the quantum vacuum is proposed via the renormalised stress-energy tensor of spacetime, and an application to the kinetically dominated universe is considered. The new theory makes potentially observable predictions.

Part~\ref{part:statistics} contains methods developed for the theoretical and observational analysis of the early universe. Chapter~\ref{chp:bay} provides an introduction to Bayesian methodologies and nested sampling.

Chapters~\ref{chp:ens}~\&~\ref{chp:pc} detail my contributions to the field of nested sampling, and have been published in~\cite{polychordletter,polychordpaper}. These demonstrate the effectiveness of the Bayesian \PolyChord{} algorithm, a novel nested sampling methodology utilising slice sampling and semi-independent posterior mode analysis.

Chapter~\ref{chp:RK} demonstrates a new method for the efficient numerical solution of oscillatory ordinary differential equations, termed the Runge-Kutta-Wentzel-Kramers-Brillouin method, and has been submitted to the Journal of Mathematical Physics.

At the end of each part, conclusions are given along with the direction of ongoing and potential further research.
\end{abstract}



