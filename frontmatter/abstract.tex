\abstractnum{}   % Format the title like a chapter title

%@TODO Anthony: ``abstract normally longer, and contains something about the results''

\begin{abstract}
This thesis is concerned with the initial conditions for inflation, and the construction of methods to aid in the observational and theoretical analysis of the early universe.
Chapter~\ref{chp:out} outlines the context and content of the thesis. After this, the thesis is divided into two parts.
The first chapters of each part (Chapters~\ref{chp:cos} and~\ref{chp:bay}) are introductory, establishing basic theory and notation. The remainder of the thesis is entirely my own work, except where references explicitly state otherwise.

Part~\ref{part:cosmology} contains theoretical and observational work in early-universe cosmology, and is divided into four chapters. Chapter~\ref{chp:cos} introduces the inflationary cosmological theory relevant to this thesis. Chapter~\ref{chp:kd} is published as ``Kinetic initial conditions for inflation'' by~\cite{Handley+2014}. Chapter~\ref{chp:kt} details further theoretical observations into the kinetically dominated universe. Chapter~\ref{chp:rec} was published as work as part of the~\cite{planck2015-a24}. Chapter~\ref{chp:qv} details theoretical work in the quantum mechanics of the early universe, and has been submitted for publication in Physical Review.

Part~\ref{part:statistics} contains methods developed for the theoretical and observational analysis of the early universe. Chapter~\ref{chp:bay} provides an introduction to Bayesian methodologies and nested sampling.
Chapters~\ref{chp:ens}~\&~\ref{chp:pc} detail my contributions to the field of nested sampling, and have been published as~\cite{polychordletter,polychordpaper}.
Chapter~\ref{chp:RK} demonstrates a new method for the numerical solution of oscillatory ordinary differential equations, and has been submitted to the Journal of Mathematical Physics.
\end{abstract}



